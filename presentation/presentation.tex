\documentclass[aspectratio=169,notes]{beamer}
\usepackage{lmodern}
\usepackage{adjustbox}
\usepackage[T1]{fontenc}
\usepackage{textcomp}
\usepackage{animate}
\usepackage{underscore}
\usepackage{pdfpc-commands}
\usepackage{xmpmulti}
\usepackage{multimedia}
\usepackage{epstopdf}
\usepackage{bbding}
\usepackage[absolute,overlay]{textpos}
\usepackage[most]{tcolorbox}

\definecolor{Title}{rgb}{0.94,0.52,0.08}
\setbeamercolor{frametitle}{bg=Title,fg=black}

% footnote without number
\makeatletter
\def\blfootnote{\xdef\@thefnmark{}\@footnotetext}
\makeatother

\usepackage{hyperref}
\usepackage{scalerel}
\def\thumbup{\scalerel*{\includegraphics{thumbup.png}}{O}}
\usepackage{listings}
\lstdefinelanguage{D}
{
  % list of keywords
  morekeywords={ abstract, alias, align, asm, assert, auto, body, bool, break,
	byte, case, cast, catch, cdouble, cent, cfloat, char, class, const,
	continue, default, double, else, enum, export, extern, false, final, finally,
	float, for, foreach, foreach_reverse, function, goto, idouble, if, ifloat,
	immutable, import, in, inout, int, interface, invariant, ireal, is, lazy,
	long, mixin, module, new, nothrow, null, out, override, package, pragma,
	private, protected, public, pure, real, ref, return, scope, shared, short,
	static, string, struct, super, switch, synchronized, template, this, throw,
	true, try, typeid, typeof, ubyte, ucent, uint, ulong, union,
	unittest, delegate, @safe, @property
	ushort, version, void, wchar, while, with, __FILE__, __FILE_FULL_PATH__,
	__MODULE__, __LINE__, __FUNCTION__, __PRETTY_FUNCTION__, __gshared,
	__traits, __vector, __parameters
  },
  otherkeywords= { @property, @safe },
  sensitive=false, % keywords are not case-sensitive
  morecomment=[l]{//}, % l is for line comment
  morecomment=[s]{/*}{*/}, % s is for start and end delimiter
  morecomment=[s]{/+}{+/}, % s is for start and end delimiter
  morestring=[b]{"}, % defines that strings are enclosed in double quotes
  morestring=[b]{`} % defines that strings are enclosed in double quotes
}
\usepackage{color}
\definecolor{eclipseBlue}{RGB}{42,0.0,255}
\definecolor{eclipseGreen}{RGB}{63,127,95}
\definecolor{eclipsePurple}{RGB}{127,0,85}

% Set Language
\lstset{
  language={D},
  basicstyle=\small\ttfamily, % Global Code Style
  captionpos=b, % Position of the Caption (t for top, b for bottom)
  extendedchars=true, % Allows 256 instead of 128 ASCII characters
  tabsize=2, % number of spaces indented when discovering a tab 
  columns=fixed, % make all characters equal width
  keepspaces=true, % does not ignore spaces to fit width, convert tabs to spaces
  showstringspaces=false, % lets spaces in strings appear as real spaces
  breaklines=true, % wrap lines if they don't fit
  numbers=left, % show line numbers at the left
  numberstyle=\tiny\ttfamily, % style of the line numbers
  commentstyle=\color{eclipseGreen}, % style of comments
  keywordstyle=\color{eclipsePurple}, % style of keywords
  stringstyle=\color{eclipseBlue}, % style of strings
}
\definecolor{lightgray}{rgb}{.9,.9,.9}
\definecolor{darkgray}{rgb}{.4,.4,.4}
\definecolor{purple}{rgb}{0.65, 0.12, 0.82}
\lstdefinelanguage{TypeScript}{
	keywords={break, case, catch, continue, debugger, default, delete, do, else,
		false, from, finally, for, function, if, in, instanceof, new, null, return, switch,
		this, throw, true, try, typeof, var, void, while, with, interface,
		class, export, boolean, throw, implements, import, this, const, let,
		of, =>},
	morecomment=[l]{//},
	morecomment=[s]{/*}{*/},
	morestring=[b]',
	morestring=[b]",
	ndkeywords={},
	keywordstyle=\color{blue}\bfseries,
	ndkeywordstyle=\color{darkgray}\bfseries,
	identifierstyle=\color{black},
	commentstyle=\color{purple}\ttfamily,
	stringstyle=\color{red}\ttfamily,
	sensitive=true
}

\colorlet{punct}{red!60!black}
\definecolor{background}{HTML}{EEEEEE}
\definecolor{delim}{RGB}{20,105,176}
\colorlet{numb}{magenta!60!black}

\lstdefinelanguage{GraphQL}{
    basicstyle=\normalfont\ttfamily,
    numbers=left,
    stepnumber=1,
    showstringspaces=false,
    breaklines=true,
	keywords={type, schema, mutation, subscription, __type, __schema, kind,
		on, fragment, query},
    literate=
     *{0}{{{\color{numb}0}}}{1}
      {1}{{{\color{numb}1}}}{1}
      {2}{{{\color{numb}2}}}{1}
      {3}{{{\color{numb}3}}}{1}
      {4}{{{\color{numb}4}}}{1}
      {5}{{{\color{numb}5}}}{1}
      {6}{{{\color{numb}6}}}{1}
      {7}{{{\color{numb}7}}}{1}
      {8}{{{\color{numb}8}}}{1}
      {9}{{{\color{numb}9}}}{1}
      {:}{{{\color{punct}{:}}}}{1}
      {,}{{{\color{punct}{,}}}}{1}
      {\{}{{{\color{delim}{\{}}}}{1}
      {\}}{{{\color{delim}{\}}}}}{1}
      {[}{{{\color{delim}{[}}}}{1}
      {]}{{{\color{delim}{]}}}}{1},
}

\lstdefinelanguage{json}{
    basicstyle=\normalfont\ttfamily,
    numbers=left,
    stepnumber=1,
    showstringspaces=false,
    breaklines=true,
    literate=
     *{0}{{{\color{numb}0}}}{1}
      {1}{{{\color{numb}1}}}{1}
      {2}{{{\color{numb}2}}}{1}
      {3}{{{\color{numb}3}}}{1}
      {4}{{{\color{numb}4}}}{1}
      {5}{{{\color{numb}5}}}{1}
      {6}{{{\color{numb}6}}}{1}
      {7}{{{\color{numb}7}}}{1}
      {8}{{{\color{numb}8}}}{1}
      {9}{{{\color{numb}9}}}{1}
      {:}{{{\color{punct}{:}}}}{1}
      {,}{{{\color{punct}{,}}}}{1}
      {\{}{{{\color{delim}{\{}}}}{1}
      {\}}{{{\color{delim}{\}}}}}{1}
      {[}{{{\color{delim}{[}}}}{1}
      {]}{{{\color{delim}{]}}}}{1},
}
\usepackage{tikz}
\usetikzlibrary{shadows,calc}
\usepackage{xkeyval}
\usepackage{todonotes}
\presetkeys{todonotes}{inline}{}
\defbeamertemplate{description item}{align left}{\insertdescriptionitem\hfill}
\usetheme{metropolis}					 % Use metropolis theme
\usepackage[
    backend=biber,
	sorting=none,
    url=true 
]{biblatex}
\addbibresource{biblio.bib}
\setbeamertemplate{bibliography item}{\insertbiblabel}

\title{Bugzilla to GitHub via Phobos --- A Love Letter}
\date{\today}
\author{Dr. Robert Schadek}
\begin{document}
	\maketitle

	\begin{frame}[fragile]{Bugzilla to GitHub via Phobos --- A Love Letter}
		\begin{center}
			{\huge D needs more people}
		\end{center}
	\end{frame}

	\section{Bugzilla to Github}
	\begin{frame}[fragile]{Bugzilla to Github}
		\Large
		\begin{itemize}
			\item Github has 83 million users
			\item issue.dlang.org has maybe 12 users
		\end{itemize}
	\end{frame}

	\begin{frame}[fragile]{The approach}
		\begin{enumerate}
			\item Get open issues from bugzilla
			\item Get issue data via rest api
			\item Find issues participates on github
			\item A bit of markdown formatting
			\item Push to github via graphql api
		\end{enumerate}
	\end{frame}

	\section{THIS IS REALLY REALLY BORING -- WE ARE NOT GOING TO TALK ABOUT THIS}

	\begin{frame}[fragile]{The interesting find}
		\begin{itemize}
			\item the header of all ten files in the project look the same
		\end{itemize}
	\end{frame}
	\begin{frame}[fragile]{source/analysis.d}
		\lstinputlisting[language=D,firstline=3,lastline=16,basicstyle=\footnotesize\ttfamily]{../source/analysis.d}
	\end{frame}
	\begin{frame}[fragile]{source/app.d}
		\lstinputlisting[language=D,firstline=1,lastline=16,basicstyle=\footnotesize\ttfamily]{../source/app.d}
	\end{frame}
	\begin{frame}[fragile]{source/getopenissues.d}
		\lstinputlisting[language=D,firstline=3,lastline=15,basicstyle=\footnotesize\ttfamily]{../source/getopenissues.d}
	\end{frame}
	\begin{frame}[fragile]{source/github.d}
		\lstinputlisting[language=D,firstline=3,lastline=17,basicstyle=\footnotesize\ttfamily]{../source/github.d}
	\end{frame}
	\begin{frame}[fragile]{source/graphql.d}
		\lstinputlisting[language=D,firstline=3,lastline=13,basicstyle=\footnotesize\ttfamily]{../source/graphql.d}
	\end{frame}
	\begin{frame}[fragile]{source/json.d}
		\lstinputlisting[language=D,firstline=3,lastline=14,basicstyle=\footnotesize\ttfamily]{../source/json.d}
	\end{frame}
	\begin{frame}[fragile]{source/markdown.d}
		\lstinputlisting[language=D,firstline=3,lastline=11,basicstyle=\footnotesize\ttfamily]{../source/markdown.d}
	\end{frame}
	\begin{frame}[fragile]{source/rest.d}
		\lstinputlisting[language=D,firstline=3,lastline=17,basicstyle=\footnotesize\ttfamily]{../source/rest.d}
	\end{frame}
	\section{All the work}
	\begin{frame}[fragile]{Parsing (the good)}
		\lstinputlisting[language=D,firstline=206,lastline=211,basicstyle=\footnotesize\ttfamily]{../source/rest.d}
		\begin{onlyenv}<1>
			\vspace{4cm}
		\end{onlyenv}
		\begin{onlyenv}<2>
\begin{lstlisting}[basicstyle=\tiny\ttfamily,numbers=none]
{"bugs":[{"priority":"P1","assigned_to_detail":{"email":"nobody","real_name":"No Owner","name":"nobody","id":606},"blocks":[],"creator":"black80","last_change_time":"2019-08-04T15:29:18Z","is_cc_accessible":true,"keywords":[],"creator_detail":{"email":"black80","real_name":"a11e99z","name":"black80","id":2335},"cc":["r.sagitario"],"url":"","assigned_to":"nobody","groups":[],"see_also":[],"id":20005,"whiteboard":"","creation_time":"2019-06-25T14:10:08Z","qa_contact":"","depends_on":[],"dupe_of":null,"resolution":"FIXED","classification":"Unclassified","alias":[],"op_sys":"Windows","status":"RESOLVED","cc_detail":[{"email":"r.sagitario","real_name":"Rainer Schuetze","name":"r.sagitario","id":648}],"summary":"VC++ can exists in separate BuildTools folder (not only in Community\\Enterprise)","is_open":false,"platform":"x86_64","severity":"enhancement","flags":[],"version":"D2","deadline":null,"component":"visuald","is_creator_accessible":true,"product":"D","is_confirmed":true,"target_milestone":"---"}],"faults":[]}
		\end{lstlisting}
		\end{onlyenv}
	\end{frame}
	\begin{frame}[fragile]{Parsing (the good)}
			\lstinputlisting[language=D,firstline=40,lastline=55,basicstyle=\footnotesize\ttfamily]{../source/rest.d}
	\end{frame}
	\begin{frame}[fragile]{Parsing (the good)}
		\begin{onlyenv}<1,3,5,7,9>
			\lstinputlisting[language=D,firstline=18,lastline=29,basicstyle=\footnotesize\ttfamily]{../source/json.d}
		\end{onlyenv}
		\begin{onlyenv}<2>
			\lstinputlisting[language=D,basicstyle=\footnotesize\ttfamily]{traits.d}
		\end{onlyenv}
		\begin{onlyenv}<4>
			\lstinputlisting[language=D,basicstyle=\footnotesize\ttfamily]{map.d}
		\end{onlyenv}
		\begin{onlyenv}<6>
			\lstinputlisting[language=D,basicstyle=\footnotesize\ttfamily]{filter.d}
		\end{onlyenv}
		\begin{onlyenv}<8>
			\lstinputlisting[language=D,basicstyle=\footnotesize\ttfamily]{filter.d}
		\end{onlyenv}
		\begin{onlyenv}<10>
			\begin{lstlisting}
import std.json;
			\end{lstlisting}
		\end{onlyenv}
	\end{frame}
	\begin{frame}[fragile]{Parsing (the good)}
		\begin{onlyenv}<1,3,5>
			\lstinputlisting[language=D,firstnumber=13,firstline=30,lastline=43,basicstyle=\footnotesize\ttfamily]{../source/json.d}
		\end{onlyenv}
		\begin{onlyenv}<2>
			\lstinputlisting[language=D,basicstyle=\footnotesize\ttfamily]{enforce.d}
		\end{onlyenv}
		\begin{onlyenv}<4>
			\lstinputlisting[language=D,basicstyle=\footnotesize\ttfamily]{fromiso.d}
		\end{onlyenv}
	\end{frame}
	\begin{frame}[fragile]{Parsing (the good)}
		\lstinputlisting[language=D,firstnumber=27,firstline=44,lastline=59,basicstyle=\footnotesize\ttfamily]{../source/json.d}
	\end{frame}
	\begin{frame}[fragile]{Parsing (the good)}
		\lstinputlisting[language=D,firstnumber=43,firstline=60,lastline=76,basicstyle=\footnotesize\ttfamily]{../source/json.d}
	\end{frame}
	\begin{frame}[fragile]{Parsing (the ugly)}
		\begin{lstlisting}[language=html,basicstyle=\tiny\ttfamily]
  <tr id="b22800" class="bz_bugitem bz_normal bz_P1 bz_NEW bz_row_odd">
    <td class="first-child bz_id_column">
      <a href="show_bug.cgi?id=22800">22800</a>
      <span class="bz_default_hidden"></span>
    </td>
    <td class="bz_product_column nowrap">
        <span title="D">D
        </span>
    </td>
    <td class="bz_component_column nowrap">
        <span title="phobos">phobos
        </span>
    </td>
    <td class="bz_assigned_to_column nowrap">
        <span title="nobody">nobody
        </span>
    </td>
    <td class="bz_bug_status_column nowrap">
        <span title="NEW">NEW
        </span>
    </td>
    <td class="bz_resolution_column nowrap">
        <span title="---">---
        </span>
    </td>
    <td class="bz_short_desc_column">
        <a href="show_bug.cgi?id=22800">DDOC throw section for writeln is incomplete        </a>
    </td>
    <td class="bz_changeddate_column nowrap">2022-02-21
    </td>
  </tr>
		\end{lstlisting}
	\end{frame}
	\begin{frame}[fragile]{Parsing (the ugly)}
		\lstinputlisting[language=D,firstnumber=1,firstline=24,lastline=37,basicstyle=\footnotesize\ttfamily]{../source/getopenissues.d}
	\end{frame}
	\begin{frame}[fragile]{Parsing (the ugly)}
		\begin{onlyenv}<1,3>
			\lstinputlisting[language=D,firstnumber=15,firstline=38,lastline=43,basicstyle=\footnotesize\ttfamily]{../source/getopenissues.d}
		\end{onlyenv}
		\begin{onlyenv}<2>
			\lstinputlisting[language=D,basicstyle=\footnotesize\ttfamily]{zip.d}
		\end{onlyenv}
		\begin{onlyenv}<4>
			\lstinputlisting[language=D,basicstyle=\footnotesize\ttfamily]{uniq.d}
		\end{onlyenv}
	\end{frame}
	\begin{frame}[fragile]{Parsing (the ugly)}
		\begin{onlyenv}<1,3,5>
			\lstinputlisting[language=D,firstnumber=21,firstline=45,lastline=58,basicstyle=\footnotesize\ttfamily]{../source/getopenissues.d}
		\end{onlyenv}
		\begin{onlyenv}<2>
			\lstinputlisting[language=D,basicstyle=\footnotesize\ttfamily]{find.d}
		\end{onlyenv}
		\begin{onlyenv}<4>
			\lstinputlisting[language=D,basicstyle=\footnotesize\ttfamily]{until.d}
		\end{onlyenv}
		\begin{onlyenv}<6>
			\lstinputlisting[language=D,basicstyle=\footnotesize\ttfamily]{to.d}
		\end{onlyenv}
	\end{frame}
	\begin{frame}[fragile]{Parsing (the ugly)}
At this point I was think to redo splitIds without phobos, then I saw I missed to
talk about regex.
\onslide<2>{
... but the phobos cheat sheet was nearly done
}
	\end{frame}

	\begin{frame}[fragile]{std.format}
		\lstinputlisting[language=D,basicstyle=\footnotesize\ttfamily]{format.d}
	\end{frame}

	\begin{frame}[fragile]{std.algorithm.searching}
		\begin{itemize}
			\item \lstinline@find@
			\begin{itemize}
				\item \lstinline@canFind@
				\item \lstinline@countUntil@
			\end{itemize}
			\item \lstinline@startsWith@
			\item \lstinline@endsWith@
		\end{itemize}
	\end{frame}

	\section{Not everything is Roses and Rainbows}

	\begin{frame}[fragile]{Common Complaints}
		\begin{itemize}
			\item GC
\uncover<2->{
			\begin{itemize}
				\item Humankind is not smart enough for manual memory management, just look at all the CVEs related to buffer overflows and memory corruptions
			\end{itemize}
}
			\item Exceptions
\uncover<3->{
			\begin{itemize}
				\item Unexpected things happen \lstinline@SysTime SysTime.fromISOExtString(string);@
				\item \lstinline@Nullable!SysTime@ is \includegraphics[width=3mm]{poop.png}
				\item \lstinline@Result!(SysTime,Error)@ is \includegraphics[width=3mm]{poop.png}
				\item \lstinline@errno@ is \includegraphics[width=3mm]{poop.png}
				\item did some say \lstinline@throw new Exception("not an ISO date")@
			\end{itemize}
}
			\item too much coupling
\uncover<4->{
			\begin{itemize}
				\item no global state shared
				\item no reason to re-implement \lstinline@find@ over and over again
				\item most of phobos functions are \lstinline@pure@
				\item this argument is just a red herring
			\end{itemize}
}
			\item auto decoding
\uncover<5->{
			\begin{itemize}
				\item decision between default incorrectness and scapegoating
			\end{itemize}
}
		\end{itemize}
	\end{frame}

	\begin{frame}[fragile]{To few things}
		\begin{itemize}
			\item no html
			\item no xml
			\item no yaml
			\item no SI units
			\item no eventcore
			\item no dmd frontend (I want CT D parsing)
			\item too little coupling
			\item why is numpy/keras/tensorflow not in phobos
		\end{itemize}
	\end{frame}

	\section{Conclusion}
	\begin{frame}[fragile]{Back to the Topic}
		\begin{itemize}
			\item To get more things, we need more people
			\item We can not tell people\\[1cm]
			\pause
			\item {\Huge We can ask people to contribute}\\[1cm]
			\item But only if they can find what we would like them to contribute too
			\item We need to ask where people are, that place is github not bugzilla
		\end{itemize}
	\end{frame}
	\begin{frame}[fragile]{Conclusion}
		\Huge
		\begin{center}
		phobos everywhere\\[1cm]
		bugzilla $\rightarrow$ github\\[1cm]
		\end{center}
	\end{frame}
	\section{Questions?}
\end{document}
